\section{Introduction}
\label{sec:introduction}
In a Sybil attack, an attacker controls many virtual identities to obtain
disproportionately large influence in a network.  These attacks take many
shapes, such as sockpuppets hijacking online discourse~\cite{Thomas2012a}; the
manipulation of BitTorrent's distributed hash table~\cite{Wang2012a}; and, most
relevant to our work, relays in the Tor network that seek to deanonymize
users~\cite{cmucert}.  In addition to coining the term ``Sybil,'' Douceur showed
that practical Sybil defenses are challenging, arguing that Sybil attacks are
always possible without a central authority~\cite{Douceur2002a}.  In this work,
we focus on Sybils in Tor---relays that are controlled by a single operator.
But what harm can Sybils do?

The effectiveness of many attacks on Tor depends on how large a fraction of the
network's traffic---the consensus weight---an attacker can observe.  As the
attacker's consensus weight grows, the following attacks become easier.

\begin{description}[noitemsep]
	\item[Exit traffic tampering:] A Tor user's traffic traverses exit relays,
		the last hop in a Tor circuit, when leaving the Tor network.
		Controlling exit relays, an attacker can sniff traffic to collect
		unencrypted credentials, break into TLS-protected connections, or inject
		malicious content~\cite{Winter2014a}.
	\item[Website fingerprinting:] Tor's encryption prevents guard relays (the
		first hop in a Tor circuit) from learning their user's online activity.
		Ignoring the encrypted payload, an attacker can still take advantage of
		flow information such as packet lengths and timings to infer what web
		site her users are connecting to~\cite{Juarez2014a}.
	\item[Bridge address harvesting:] Users behind censorship firewalls use
		private Tor relays (``bridges'') as hidden stepping stones into the Tor
		network.  It is important that censors cannot obtain all bridge
		addresses, which is why bridge distribution is rate-limited.  However,
		an attacker can harvest bridge addresses by running a middle relay and
		looking for incoming connections that do not originate from any of the
		publicly known guard relays~\cite{Ling2015b}.
	\item[End-to-end correlation:] By running both entry guards and exit relays,
		an attacker can use timing analysis to link a Tor user's identity to her
		activity, e.g., learn that \emph{Alice} is visiting \emph{Facebook}.
		For this attack to work, an attacker must run at least two Tor relays,
		or be able to eavesdrop on at least two networks~\cite{Johnson2013a}.
\end{description}

Configuring a relay to forward more traffic allows an attacker to increase her
consensus weight.  However, the capacity of a single relay is limited by its
link bandwidth and, because of the computational cost of cryptography, by CPU.
Ultimately, increasing consensus weight requires an adversary to add relays to
the network; we call these additional relays Sybils.

In addition to the above attacks, an adversary needs Sybil relays to manipulate
onion services, which are TCP servers whose IP address is hidden by Tor.  In the
current onion service protocol, six Sybil relays are sufficient to take offline
an onion service because of a weakness in the design of the distributed hash
table (DHT) that powers onion services~\cite{Biryukov2013a}.  Finally, instead
of being a direct means to an end, Sybil relays can be a \emph{side effect} of
another issue.  In Section~\ref{sec:sybil_groups}, we provide evidence for what
appears to be botnets whose zombies are running Tor relays, perhaps because of a
misguided attempt to help the Tor network grow.

Motivated by the lack of practical Sybil detection tools, we design and
implement heuristics, leveraging that Sybils (\emph{i}) frequently go online and
offline simultaneously, (\emph{ii}) share similarities in their configuration,
and (\emph{iii}) may change their identity fingerprint---a relay's fingerprint
is the hash over its public key---frequently, to manipulate Tor's DHT.  We
implemented these heuristics in a tool, \sys, whose development required a major
engineering effort because we had to process 100 GiB of data and millions of
files.  We used \sys to analyze archived network data, dating back to 2007, to
discover past attacks and anomalies.  Finally, we characterize the Sybil groups
we discovered.  To sum up, we make the following key contributions:
\begin{itemize}
	\item We design and implement \sys, a tool to analyze past and future Tor
		network data.  While we designed it specifically for the use in Tor, our
		techniques are general in nature and can easily be applied to other
		distributed systems such as I2P~\cite{i2p}.
	\item We expose and characterize Sybil groups, and publish our findings as
		datasets to stimulate future research.\footnote{The dataset is online at
			\url{https://nymity.ch/sybilhunting/}.} We find that Sybils run MitM
		attacks, DoS attacks, and are used for research projects.
\end{itemize}

The rest of this paper is structured as follows.  We begin by discussing related
work in Section~\ref{sec:related_work} and give some background on Tor in
Section~\ref{sec:background}.  Section~\ref{sec:design} presents the design of
our analysis tools, which is then followed by experimental results in
Section~\ref{sec:results}.  We discuss our results in
Section~\ref{sec:discussion} and conclude the paper in
Section~\ref{sec:conclusion}.
