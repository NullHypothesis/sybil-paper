\section{Introduction}
\label{sec:introduction}

% Short introduction to Sybil attacks.
Distributed systems are believed to be more robust than their centralized
counterparts against attacks such as coercion, denial of service, and
surveillance.  The Tor network is no exception.  However, the absence of a
central, controlling authority facilitates \emph{Sybil attacks}.  Sybil
attacks---coined by Douceur in 2002~\cite{Douceur2002a}---are defined as one
physical entity controlling many virtual identities in a distributed system.
The attack is problematic because many distributed systems make assumptions on
the maximum number of malicious nodes they can endure.  Indeed, the attack is
not only of academic interest.  In 2014, researchers added Sybils to the Tor
network to deanonymize onion service users~\cite{cmucert}.

% How do Sybil attacks relate to Tor? -- DHT manipulation.
An attacker's motivation for performing a Sybil attack depends on the system she
is targeting.  In the case of Tor, attackers are frequently motivated by
manipulating Tor's distributed hash table (DHT)~\cite{rendspec} that is used to
implement onion services.  Tor clients need an onion service descriptor to learn
how to connect to an onion service such as 3g2upl4pq6kufc4m.onion, the
DuckDuckGo search engine.  These descriptors are stored in Tor's DHT.  The DHT
implementation suffers from a vulnerability that allows an attacker to predict
at which position in the DHT a descriptor will be stored at some point in the
future.  The attacker can then change its fingerprints to become responsible for
a given onion service~\cite{Biryukov2013a}.  Six Sybil relays are sufficient to
become the sole responsible party for an onion service, allowing the attacker to
monitor (still anonymous) clients requests, or refuse to serve the onion service
descriptor, effectively censoring it.

% How do Sybil attacks relate to Tor? -- Increased traffic exposure.
In addition to DHT manipulation, Sybils in Tor allow an attacker to increase its
exposure to traffic.  Because of Tor's bandwidth-weighted relay selection,
running $n$ relays equates to running a single relay that provides as much
bandwidth as all $n$ relays.  But at some point, an attacker will have to scale
horizontally because there are limits to how much traffic can be relayed by a
single machine.  As an attacker's bandwidth grows, she is increasingly able to:
\begin{description}
	\item[Sniff and manipulate exit traffic] to steal credentials, break into
		TLS connections, or inject data~\cite{Winter2014a}.
	\item[End-to-end correlate] traffic by running entry guards as well as exit
		relays~\cite{Johnson2013a}.  Note that large network-level adversaries
		such as an ISP or a government might only need to run one of both.
	\item[Harvest bridge addresses] by running middle nodes and isolating the
		IP addresses of incoming Tor connections that don't originate from
		(publicly known) guard relays~\cite{Ling2012a}.
	\item[Website fingerprint] connections on guard relays to learn what
		websites Tor users are connecting to~\cite{Juarez2014a}.
\end{description}

% Sybils can be a side effect.
Instead of being a direct means to an end, Sybil relays can be a \emph{side
effect} of another issue.  In Section~\ref{sec:sybil_groups}, we provide
evidence for what appears to be botnets whose zombies are running Tor relays.
In this case, the relay operators are probably not aware of the relays they are
running.

% Existing Sybil defense mechanisms.
Sybil attacks are a well-known problem to The Tor Project, which is reflected in
a number of implicit and explicit Sybil defenses that are in place as of Nov
2015.  First, directory authorities accept at most two relays per IP
address~\cite{Bauer2007b} to prevent low-resource Sybil
attacks~\cite{Bauer2007a}.  Similarly, the path selection
algorithm~\cite{path-spec} contains heuristics to avoid relays that could be
controlled by the same operator, e.g., Tor clients never select two relays in
the same /16 network.  Second, The Tor Project has recently increased the
minimal time until relays obtain the \texttt{Stable} flag (seven days) and the
\texttt{HSDir} (96 hours) flag, to increase the cost of Sybil relays and to give
Tor developers more time to discover suspicious relays before they are in a
position to run an attack.  Finally, operating a Tor relay causes recurring
costs---most notably bandwidth and CPU---which can further limit an adversary.

% Our contributions.
Motivated by the lack of practical Sybil detection tools, we design and
implement techniques to find Sybils in the Tor network.  We implemented these
techniques in a tool, sybilhunter, which we then use to analyze historical
network data, dating back to as early as 2007, to discover past attacks and
anomalies.  Finally, we characterize the Sybil groups we discover.  To sum up,
we make the following key contributions:
\begin{itemize}
	\item We design and implement a tool---sybilhunter---to analyze past and
		future Tor data.  Our approach does not require any modifications to the
		Tor network.
	\item We expose and characterize Sybil clusters and publish a dataset to
		stimulate future research.
\end{itemize}

% Structure of the paper.
The rest of this paper is structured as follows.  We begin by discussing
related work in Section~\ref{sec:related_work}.  Section~\ref{sec:design}
presents the design of our analysis tools, which is then followed by
experimental results in Section~\ref{sec:results}.  We discuss our results in
Section~\ref{sec:discussion} and conclude the paper in
Section~\ref{sec:conclusion}.
