\section{Introduction}
\label{sec:introduction}

% Short introduction to Sybil attacks.
In a Sybil attack, one physical entity controls many virtual identities to
obtain disproportionately large influence in a network.  These attacks take many
shapes, such as sockpuppets hijacking online discourse~\cite{Thomas2012a}; the
manipulation of BitTorrent's distributed hash table~\cite{Wang2012a}; and, most
relevant to our work, relays in the Tor network that seek to deanonymize
users~\cite{cmucert}.  Preventing Sybil attacks remains challenging, as
evidenced by Douceur's seminal 2002 paper that shows that Sybil attacks are
always possible in the absence of a central authority~\cite{Douceur2002a}.  In
this work, we focus on Sybils in the Tor network, i.e., Tor relays (virtual
identities) that are controlled by a single operator (physical entity).  But
what harm can Sybils do in Tor?

The effectiveness of many attacks on Tor depends on how large a fraction of the
network's traffic---the consensus weight---an attacker can observe.  As the
attacker's consensus weight grows, she is increasingly able to launch attacks
such as:
\begin{description}
	\item[Exit relay tampering:] A Tor user's traffic traverses exit relays, the
		last hop in a Tor circuit, when leaving the Tor network.  Having control
		over exit relays, an attacker can sniff traffic to harvest unencrypted
		credentials, break into TLS-protected connections, or inject traffic
		such as advertisements~\cite{Winter2014a}.
	\item[Website fingerprinting:] Tor's encryption prevents guard relays (the
		first hop in a Tor circuit) from learning their user's online activity.
		Ignoring the encrypted payload, an attacker can still take advantage of
		flow information such as packet lengths and timings to infer what web
		site its users are connecting to~\cite{Juarez2014a}.
	\item[Bridge address harvesting:] Bridges are unpublished Tor relays that
		are given to censored users as hidden stepping stone into the Tor
		network.  It is important that censors cannot obtain all bridge
		addresses, which is why bridge distribution is rate-limited.  An
		attacker can harvest bridge addresses by running a middle relay and
		looking for incoming connections that do not originate from guard
		relays.  These have to be bridge addresses~\cite{Ling2012a}.
	\item[End-to-end correlation:] By running both entry guards and exit relays,
		the attacker can use timing analysis to link a Tor user's activity to
		her identity, e.g., learn that Alice visited Facebook.  For this attack,
		an attacker must run at least two Tor relays~\cite{Johnson2013a}.
\end{description}

% Scale horizontally when we run out of consensus weight.
An attacker can increase its consensus weight by configuring her relay to
forward more traffic.  However, the capacity of a single relay is limited by its
link bandwidth and, because of the computational cost of cryptography, by CPU.
Once an adversary reaches this limit, she has to scale horizontally, i.e., add
more Sybil relays to the network.

% Sybils are needed to manipulate the DHT, and can be a side effect.
Besides the above attacks, an adversary needs Sybil relays to manipulate onion
services, which are TCP servers whose IP address is hidden by Tor.  Six Sybil
relays are sufficient to take offline an onion service because of a weakness in
the design of the distributed hash table (DHT) that powers onion
services~\cite{Biryukov2013a}.  Finally, instead of being a direct means to an
end, Sybil relays can be a \emph{side effect} of another issue.  In
Section~\ref{sec:sybil_groups}, we provide evidence for what appears to be
botnets whose zombies are running Tor relays, perhaps because of a misguided
attempt to help the Tor network grow.  The zombie owners are likely unaware of
the relays they are running.

% Existing Sybil defense mechanisms.
Sybil attacks are a well-known problem to The Tor Project, which is reflected in
a number of both implicit and explicit Sybil defenses that are in place as of
Dec 2015.  First, directory authorities---the ``gatekeepers'' of the Tor
network---accept at most two relays per IP address~\cite{Bauer2007b} to prevent
low-resource Sybil attacks~\cite{Bauer2007a}.  Similarly, Tor's path selection
algorithm~\cite{path-spec} incorporates heuristics to avoid relays that could be
controlled by the same operator, e.g., Tor clients never select two relays in
the same /16 network.  Second, directory authorities assign flags to relays,
indicating their status and quality of service.  The Tor Project has recently
increased the minimal time until relays obtain the \texttt{Stable} flag (seven
days) and the \texttt{HSDir} flag (96 hours).  This change increases the cost of
Sybil attacks and gives Tor developers more time to discover suspicious relays
before they get in a position to run an attack.  Finally, operating a Tor relay
causes recurring costs---most notably bandwidth and electricity---which can
further limit an adversary.

% Our contributions.
Motivated by the lack of practical Sybil detection tools, we design and
implement techniques to find Sybils in the Tor network.  We implemented these
techniques in a tool, sybilhunter, which we then use to analyze historical
network data, dating back to as early as 2007, to discover past attacks and
anomalies.  Finally, we characterize the Sybil groups we discover.  To sum up,
we make the following key contributions:
\begin{itemize}
	\item We design and implement a tool---sybilhunter---to analyze past and
		future Tor data.  Our approach does not require any modifications to the
		Tor network, finds similarities in relays' configuration and their
		uptime sequences.
	\item We expose and characterize Sybil clusters and publish a dataset to
		stimulate future research.  We find that many different kinds of
		operators run Sybil groups, including financially-motiviated attackers,
		researchers, and unskilled troublemakers.
\end{itemize}

% Structure of the paper.
The rest of this paper is structured as follows.  We begin by discussing
related work in Section~\ref{sec:related_work}.  Section~\ref{sec:design}
presents the design of our analysis tools, which is then followed by
experimental results in Section~\ref{sec:results}.  We discuss our results in
Section~\ref{sec:discussion} and conclude the paper in
Section~\ref{sec:conclusion}.
