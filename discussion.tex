\section{Discussion}
\label{sec:discussion}
Having presented and evaluated our techniques, we will now discuss the
transparency and secrecy tradeoff, we reiterate our work's limitations, and we
discuss the role of cloud providers in Sybil attacks.

\subsection{Balancing transparency and secrecy}
\label{sec:secrecy}
During the development of sybilhunter, we pondered what a reasonable balance
between transparency and secrecy should look like.  We want our system design
and code to be open, to stimulate scientific progress, but a freely available
implementation helps attackers evade our system because they can first test and
refine their attacks offline.

To mitigate this problem, we divide our system into the \emph{open} analysis
framework and its \emph{secret} parameters.  After all, the analysis framework
is of primary interest to other researchers, whereas its parameters are mere
operational details.  Note that the authors of exitmap follow a similar
philosophy by making available exitmap's scanning framework~\cite{exitmap}, but
sharing its modules only privately.  This differentiation seems to be
sustainable as attackers are primarily interested in scanning modules, e.g.,
which URLs, protocols, and ports are probed.  The scanning framework, which is
of primary interest to the scientific community, is available as free software.

Unfortunately, we are unable to achieve a setting analogous to Kerckhoffs'
principle in cryptography, stating that a system must be secure even if all
except the key is known.  In sybilhunter, unlike modern cryptograhpic systems,
there is a clear relationship between the input (the system parameters) and the
output (an alert).  Still, an attacker has to make an effort to stay under the
radar of our system.  Given that it is impossible to always find all Sybils, our
efforts are ultimately limited to increasing the cost of an attack, e.g.,
forcing an attacker to infer our system's parameters.

\subsection{Limitations}
\label{sec:limitations}
% We cannot detect all Sybils.  We can only make an effort.
We argued in Section~\ref{sec:threat_model} that we are unable to prevent all
Sybil attacks.  An adversary unconstrained by time and money will always be able
to add Sybils and eliminate redundancy in her Sybils' appearance and behavior.
This is known since Douceur showed in 2002 that the only way to prevent Sybil
attacks is a central authority that verifies network
participants~\cite{Douceur2002a}.  A central authority is unlikely to be viable
for the Tor network.  It would be in conflict with Tor's goal of distributing
trust and alienate relay operators.  As a result, detecting Sybils in the Tor
network is limited to making a best effort.

% We cannot determine the purpose of a Sybil group.
Finally, our approach is unable to determine the \emph{purpose} of a Sybil
attack.  In some cases, the purpose is obvious; for example, if a Sybil group
has the \texttt{Exit} flag and tampers with traffic, or if a group has the
\texttt{HSDir} flag and shares an unusually long fingerprint prefix.  In many
other cases, however, we rely on additional information to decide if a Sybil
group should be removed.  For example, Sybils originating from ``bulletproof''
ASes~\cite{Konte2015a}, showing signs of not running the tor reference
implementation, or spoofing information in their router descriptor could all
suggest malicious intent.  In the end, Sybil groups have to be evaluated case
by case, and the benefits and disadvantages of blocking it have to be evaluated.
