\section{Discussion}
\label{sec:discussion}
Having presented and evaluated our techniques, we will now discuss the
transparency and secrecy tradeoff~(\S~\ref{sec:secrecy}), we reiterate our
work's limitations~(\S~\ref{sec:limitations}), and we discuss the role of cloud
providers in Sybil attacks~(\S~\ref{sec:cloud}).

\subsection{Balancing transparency and secrecy}
\label{sec:secrecy}
During the development of sybilhunter, we pondered what a reasonable balance
between transparency and secrecy should look like.  On the one hand, we want our
system design and code to be open, to stimulate scientific progress.  On the
other hand, a freely available implementation helps attackers evade our system
because they can first test and refine their attacks offline.

It seems difficult to achieve a setting analogous to Kerckhoffs' principle in
cryptography, stating that a system must be secure even if everything except the
key is known about it.  There is no key in our setting.  We can, however, divide
our system into the \emph{open} analysis framework and its \emph{secret}
parameters.  After all, the analysis framework is of primary interest to other
researchers, whereas its parameters are mere operational details.

Note that the authors of exitmap follow a similar philosophy by making available
exitmap's scanning framework~\cite{exitmap}, but sharing its modules only
privately.  This differentiation seems to be sustainable as attackers are
primarily interested in scanning modules, e.g., which URLs, protocols, and ports
are probed.

\subsection{Limitations}
\label{sec:limitations}
% We cannot detect all Sybils.  We can only make an effort.
We mentioned in Section~\ref{sec:threat_model} that we are unable to prevent all
Sybil attacks.  An adversary unconstrained by time and money will always be able
to add Sybils and eliminate redundancy in her Sybils' appearance and behavior.
This is known since Douceur showed in 2002 that the only way to prevent Sybil
attacks is a central authority that verifies network
participants~\cite{Douceur2002a}.  A central authority is unlikely to be viable
for the Tor network.  It would be in conflict with Tor's goal of distributing
trust and alienate relay operators.  As a result, detecting Sybils in the Tor
network is limited to making a best effort.

% We cannot determine the purpose of a Sybil cluster.
Finally, our approach is unable to determine the \emph{purpose} of a Sybil
attack.  In some cases, the purpose is obvious; for example, if a Sybil cluster
has the \texttt{Exit} flag and tampers with traffic, or if a cluster has the
\texttt{HSDir} flag and shares an unusually long fingerprint prefix.  In many
other cases, however, we rely on additional information to decide if a Sybil
cluster should be removed.  For example, Sybils originating from ``bulletproof''
ASes~\cite{Konte2015a}, showing signs of not running the tor reference
implementation, or spoofing information in their router descriptor could all
suggest malicious intent.  In the end, Sybil clusters have to be evaluated case
by case, and the benefits and disadvantages of blocking it have to be evaluated.

% We will now explore how costly a Sybil attack is in practice.  First, an
% adversary needs a number of systems to run Tor relays on.  These systems should
% be geographically distributed to maximize IP address diversity.  One option is
% to rent virtual private systems, starting at around \$3 per month for 1 Gbps.
% In 2011, the price for 1,000 compromised systems to install malware on (so
% called \emph{loads}) ranged from \$13 (in Asia) to \$125 (in the U.S.)\cite[\S
% 5]{Stone-Gross2011a}.

\subsection{Use and abuse of cloud providers}
\label{sec:cloud}
Some of the Sybil groups listed in Table~\ref{tab:sybils} were hosted in the
cloud.  Presumably, cloud-hosted relays are attractive to attackers because they
provide cheap, disposable, and hourly-billed platforms.  But does the bandwidth
contributed by cloud-hosted relays make up for the abuse?

To answer this question, we first calculated the amount of bandwidth contributed
by Tor relays that were located in the netblocks of three major cloud providers,
Amazon AWS~\cite{amazonaws}, Google Cloud Platform~\cite{googlecloud}, and
Microsoft Azure~\cite{azure}.  Note that the netblocks published by these cloud
providers can change over time, and are not archived.  As a result, we could
have missed netblocks, which means that our calculations can only provide a
lower bound of cloud-hosted bandwidth.

Since we do not have access to archived netblocks, we limit our analysis to
July 2015.  Having obtained cloud-hosted netblocks, we then iterated over all
744 consensus files from July 2015 and identified Tor relays that were hosted
by Google, Amazon, or Microsoft.  On average, 189 out of 6,540 Tor relays
(2.9\%) were run in cloud-powered IP address space.  Because Tor
clients select relays in their circuits based on bandwidth, we then determined
the fraction these cloud-hosted relays contributed to the total Tor bandwidth.
The results are shown in Table~\ref{tab:bwfraction}.  The median contributed
bandwidth is 0.8\%.  There were no Google-hosted relays.  Amazon-hosted relays
contributed about 18 times more bandwidth than Microsoft-hosted relays.

\begin{table}[t]
	\centering
	% \begin{tabular}{lllllll}
	\begin{tabular}{lllll}
	% Provider & Min. & 1st Qu. & Median & Mean & 3rd Qu. & Max. \\
	\textbf{Provider} & \textbf{Min.} & \textbf{Median} & \textbf{Mean} & \textbf{Max.} \\
	\hline
	% Google & 0 & 0 & 0 & 0 & 0 & 0 \\
	Google & 0 & 0 & 0 & 0 \\
	% Amazon & 0.2 & 0.7 & 0.7 & 0.76 & 0.8 & 1.5 \\
	Amazon & 0.2 & 0.7 & 0.76 & 1.5 \\
	% Microsoft & 0 & 0 & 0 & 0.02 & 0 & 0.1 \\
	Microsoft & 0 & 0 & 0.02 & 0.1 \\
	\hline
	% Total & 0.2 & 0.7 & 0.8 & 0.79 & 0.8 & 1.5 \\
	Total & 0.2 & 0.8 & 0.79 & 1.5 \\
	\end{tabular}
	\caption{Percentage of total Tor bandwidth in July 2015 contributed by
	relays hosted in Google's, Amazon's, or Microsoft's cloud.}
	\label{tab:bwfraction}
\end{table}

In addition to relays, the Tor network has bridges, basically unpublished Tor
relays that are used for censorship circumvention.  The IP addresses of bridges
are not published, which prevents us from repeating the bandwidth analysis.  The
Tor Project however publishes the number of bridges in the Tor
Cloud~\cite{torcloud}, a service to easily set up an EC2-powered bridge,
leveraging Amazon's free usage tier.  The number of Tor Cloud bridges,
illustrated in Figure~\ref{fig:cloudbridges}, can serve as proxy variable for
the amount of bandwidth they contribute.  The number has been decreasing over
time, and on May 8, 2015, the Tor Cloud program was shut down.  As a result, we
expect the number of cloud bridges to keep decreasing as the free usage tier of
more bridge operators runs out.

\begin{figure}[t]
	\centering
	\includegraphics[width=\linewidth]{diagrams/torcloud.pdf}
	\caption{The amount of Tor Cloud~\cite{torcloud} bridges over time.  The
	numbers are steadily decreasing and the Tor Cloud system was discontinued in
	May 2015.}
	\label{fig:cloudbridges}
\end{figure}
