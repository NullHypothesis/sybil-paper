\section{Related work}
\label{sec:related_work}
% Why previously proposed solutions don't work.
In his seminal 2002 paper~\cite{Douceur2002a}, Douceur formally showed that the
only method guaranteed to keep out Sybils is a \emph{central authority} that
verifies new nodes as they join the distributed system.  This approach conflicts
with Tor's design philosophy that seeks to eliminate central points of control,
to distribute trust.  In addition, a major factor contributing to Tor's relay
growth is the low barrier of entry, allowing operators to set up relays quickly
and anonymously.  An identity-verifying authority would raise that barrier and
thwart Tor's growth.  Barring a central authority, researchers have proposed
techniques that build on a resource that is difficult for an attacker to scale.
Two categories of Sybil-resistant schemes turned out to be particularly popular:
schemes that build on \emph{social} and schemes that build on
\emph{computational} constraints.  For a broader overview of Sybil defenses,
refer to Levine, Shields, and Margolin's technical report~\cite{Levine2006a}.

Social constraints are based on the assumptions that it is difficult for an
attacker to form trust relationships with honest users.  Past work exploited
this fact in systems such as SybilGuard~\cite{Yu2006a},
SybilLimit~\cite{Yu2008a}, and SybilInfer~\cite{Danezis2009a}.  However, social
graph-based defenses don't work in our setting because there is \emph{no
existing trust relationship} between the operators of Tor relays.  Note that we
could create such a relationship, e.g., by linking relays to their operator's
social networking account, or by creating a ``relay operator web of trust'' from
scratch, but again, we believe that such an effort would receive limited
adoption.

Orthogonal to social constraints, computational resource constraints can be
useful if an attacker that seeks to operate 100 Sybils suddenly needs 100 times
the computational resources she would need for a single virtual identity.
Previous work used computational puzzles for that
purpose~\cite{Borisov2006a,Li2012a}.  However, requiring relay operators to
complete proof-of-works is pointless because running a relay already implies
computational work, i.e., relaying data; in other words, there is no way to run
100 Tor relays while not spending the resources for 100 relays.  In summary, we
believe that existing Sybil defences don't work well when applied to the Tor
network; its destinctive features call for special solutions.

In parallel to Sybil prevention, research has focused on characterizing
real-world Sybils.  Wang and Kangasharju uncovered a Sybil attack in
BitTorrent's distributed hash table~\cite{Wang2012a}.  Thomas, Grier, and Paxson
found several thousand Sybil accounts on Twitter to dilute political
speech~\cite{Thomas2012a}, and much work has focused on detecting Sybils that
are used to send spam~\cite{Gao2010a}.
