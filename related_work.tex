\section{Related work}
\label{sec:related_work}
The concept of a Sybil attack was coined by Douceur in 2002~\cite{Douceur2002a}.
In his seminal work, Douceur analytically showed that without a central
verification authority, Sybil attacks are always possible.  The attack can be
mitigated, however, by making assumptions on the attacker's resources.  These
assumptions became the focus of subsequent work, in which researchers tried to
find resources that are difficult for an attacker to scale.

Two categories of Sybil-resistant schemes turned out to be particularly popular:
schemes that build on \emph{social} and \emph{resource} constraints.  Social
constraints are based on the insight that it is difficult for an attacker to
build trust relationships with honest users.  Past work exploited this fact in
systems such as SybilGuard~\cite{Yu2006a}, SybilLimit~\cite{Yu2008a}, and
SybilInfer~\cite{Danezis2009a}.  Resource constraints were based on
computational puzzles~\cite{Borisov2006a,Li2012a}.  For an overview of Sybil
defenses, refer to Levine, Shields, and Margolin's technical
report~\cite{Levine2006a}.  Related to the Tor network, Margolin and
Levine~\cite{Margolin2008a} evaluated a recurring fee onion routing protocol in
which network participants pay a small amount of money for circuit creation.

In parallel to Sybil prevention, research has focused on characterizing
real-world Sybils.  Wang and Kangasharju uncovered a Sybil attack in
BitTorrent's distributed hash table~\cite{Wang2012a}.  Thomas, Grier, and Paxson
found several thousand Sybil accounts on Twitter to dilute political
speech~\cite{Thomas2012a}, and much work has focused on detecting Sybils that
are used to send spam~\cite{Gao2010a}.

The rest of our work will show that previous work is difficult to apply; the
lack of trust relationships between relays frustrates social graph-based
methods and resource constraints are inherent to running a Tor relay, and hence
useless.  We draw, however, inspiration from previous work on how to detect
Sybils~\cite{Liu2015a}.
